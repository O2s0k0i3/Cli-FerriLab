% Indica la clase del documento, tiene parámetros opcionales como el tamaño de letra o el de la página.
\documentclass[10pt]{article}
% Codificación del idioma en español ya que por defecto está en inglés.
\usepackage[utf8]{inputenc}
\usepackage[spanish, es-tabla]{babel}
% Permite escribir mejor en modo matemático. Amsmath y xparse son necesarios para que funcione.
\usepackage{physics}
\usepackage{amsmath}
\usepackage{xparse}
% Permite hacer subfiguras dentro del mismo entorno figure.
\usepackage[margin=50pt]{caption}
\usepackage{subcaption}
% Permite el trabajar con figuras en latex.
\usepackage{graphicx}
\usepackage{wrapfig}
% Permite manejar colores de manera útil.
\usepackage{xcolor, colortbl}
% Define un color que puedes utilizar referenciando el nombre dado.
\definecolor{lightgray}{rgb}{.76, .76, .76}
% Permite modificar los márgenes del documento.
\usepackage[letterpaper,top=1.5cm,bottom=1.5cm,left=1.5cm,right=1cm,marginparwidth=1.75cm]{geometry}
% Hace que los links o referencias en el texto se muestren en otro color a escoger.
\usepackage[colorlinks=true, allcolors=blue]{hyperref}
% Para ordenar mejor tablas y figuras.
\usepackage{float}
% Para crear las bibliografías, cambiando el estilo al deseado.
\usepackage[backend=biber, sorting=none, style=numeric]{biblatex}
\usepackage{csquotes}
\bibliography{referencias}
% Establece un espacio entre párrafos.
\setlength{\parskip}{3mm}
% Para colocar barras diagonales en tablas.
\usepackage{slashbox}

% Comandos varios personalizados.

\newcommand{\up}[1]{\textsuperscript{#1}}
\newcommand{\down}[1]{\textsubscript{#1}}
\newcommand{\diffalign}[2]{\multicolumn{1}{#1}{#2}}
\newcommand{\espacio}{\text{ }}
\newcommand{\fig}[1]{Fig. \ref{#1}}
\newcommand{\eq}[1]{Ec. \eqref{#1}}
\newcommand{\tab}[1]{Tab. \ref{#1}}
\newcommand{\propagar}[2]{\qty(\pdv{#1}{#2} \Delta #2)^2}

\title{Title}
\author{Author}
\date{DD/MM/YYYY}
